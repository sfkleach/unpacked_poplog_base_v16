% RCS LOG
%
% $Log:	writers.tex,v $
%Revision 1.1  91/03/27  16:42:03  16:42:03  db (Dave Berry)
%Initial revision
%

\chapter{Writing Library Entries.}		\label{implement}

The library provides a consistent framework which can easily incorporate
new entries.  New entries should use the same format as the existing
entries, to keep the framework consistent.  For example, the standard
names described in Section~\ref{names} should be used where appropriate.
This chapter mentions some particular points that you should observe.

Look at the existing entries before writing your own; this may
help focus your ideas.  The library includes two skeleton files, one for
signatures and one for functors.

Don't use the identifiers {\tt import}, {\tt abstraction} and {\tt abstract}.
The first two are keywords in Standard ML of New Jersey, and the third
is proposed in an extension to the Definition of SML.  Using them will
make your code non-portable.
 

\section{Format of Signatures.}

The format of user-visible signatures is described in Chapter~\ref{entries}.  
New entries should be written in the same style, so that users can
easily find the information they want.

Some aspects of the format may not be immediately obvious:
\begin{itemize}
  \item The header section must include creation and maintenance details.
	If no-one is maintaining this entry, the maintenance line should
	be: {\tt Maintenance: None}.
  \item The header section must also include a general description of
	the entry.  The combination of this description and the comments
	associated with each function should be suitable for on-line reading.
	If more detailed documentation is needed, it should be put
	in the {\tt doc} directory.
  \item The only comments to begin in the left-most column are the header
	section and the sub-headings used to structure the entry.
  \item The comments associated with each object should immediately follow
	its specification.
\end{itemize}


\section{Format of Structures and Functors.}

A structure or functor that implements a user-visible signature should
provide creation and maintenance details in the same format.  It may also
include a general description of the implementation, but this is not
required.

Functors arguments should always be written in the form
\verb+structure foo: FOO+.  The short form, \verb+functor F (foo: FOO) = ...+,
should not be used.  Each
functor that is intended to be visible to users should have argument and
result signatures that meet the requirements of the previous section.  The
header section should come after the arguments to the functor.


\section{Installing A New Entry.}

Each entry should be described by a signature.  This will usually be a
new signature written for the entry, although some entries will be described
by existing generic signatures.  New user-visible signatures should be
placed in the {\tt signatures} sub-directory of the library.  They
should include dependency declarations for the {\tt Make} system.  Most
signatures will only depend on {\tt InstreamTypes} and {\tt GeneralTypes}.

Many entries will be simple enough to be implemented by a single structure
of functor.  In this case the structure or functor should be placed in 
a file in the {\tt portable} subdirectory (or a system-specific distribution
sub-directory if appropriate).  It should contain a dependency declaration
for the {\tt Make} system, and also an application of the {\tt loadSig}
function to a string containing the name of the user-visible signature.
The latter is used by the {\tt build\_all.sml} and {\tt build\_core.sml}
build files.

More complex entries (such as the {\tt Make} system) will comprise
several structures and functors, and more signatures than the
user-visible one.  Such entries should be given a sub-directory in
the {\tt portable} directory, containing all the files except the
user-visible signature.  In the existing entries, I've prefixed
all local structure and signature identifiers by the name of the
whole package, to make them less likely to clash with existing or
user-defined names.

The local files of a complex package should be given dependency declarations
for the {\tt Make} system, as usual.\footnote{The two existing sub-directories,
{\tt Make} and {\tt Core}, don't have dependency declarations because
they're loaded by {\tt build\_make.sml} itself.}
In addition, they should use the
{\tt loadLocalSig} and {\tt loadEntry} functions to load files in the correct
sequence when they are built without {\tt Make}.  You will need to change
the {\tt build\_all.sml}
build file to load the package properly.  The function {\tt setLoadPrefix}
must be applied to the name of the sub-directory before the call to
{\tt loadEntry} that loads the package, and applied to the empty string
afterwards.  Look at the existing code for the {\tt Make} and{\tt Core}
sub-directories.

You will also have to change the {\small\tt INSTALL} file for every
new sub-directory that you create.  The first change is to generate
the correct entries in the {\small\tt ML\_CONSULT} file used by the
{\tt build\_make.sml} build file.  The second change is to ensure that
the new entry is linked into all versions of the library.  Again, look at
the existing code for the {\tt Make} and {\tt Core} subdirectories.

After you've added a new entry, you should run the {\small\tt INSTALL}
program to generate the new {\small\tt ML\_CONSULT} file for
{\tt build\_make.sml}.

